\documentclass[a4paper]{article}
\usepackage[utf8]{inputenc}
\usepackage[T1]{fontenc}
\usepackage[brazil]{babel}
\usepackage[margin=2cm]{geometry}
\usepackage{amsmath,amsfonts,amssymb,indentfirst}
\usepackage{booktabs,graphicx,hyperref,tikz}
\usepackage{wrapfig}
\usepackage{fancyhdr}

\pagestyle{fancy}
\lhead{}
\chead{} 
\rhead{}
\lfoot{}
\cfoot{}
\rfoot{\thepage}


\title{Proposta de Trabalho de Conclusão de Curso}
\author{UTFPR - Departamento Acadêmico de Eletrônica}
\date{\the\year}

\begin{document}

\noindent
\includegraphics[width=0.15\textwidth]{figuras/UTFPR.png}
\begin{minipage}[b]{0.7\textwidth}
\centering
\Large{
Universidade Tecnológica Federal do Paraná – UTFPR\\
Departamento Acadêmico de Eletrônica\\
Curso de Engenharia Eletrônica \\
Proposta de Trabalho de Conclusão de Curso}
\end{minipage}
\includegraphics[width=0.15\textwidth]{figuras/daeln.jpg}

\section{IDENTIFICAÇÃO DO PROJETO}

Projeto Final I - Turma 2016/1 \\

Título: Relógio monitor para Idosos \\

Equipe:
\newline
	
\begin{tabular}{llll}
	1301306 & Erik Marcon & erikmarcon@alunos.utfpr.edu.br & (41) 9830-9199 \\
	0944084 & Igor Ivan Gaudeda & gaudeda@alunos.utfpr.edu.br & (41) 9990-0944
\end{tabular} \\

Professor Orientador: Daniel Rossato de Oliveira \\

Resumo: O projeto visa a implementação de um sistema completo de monitoramento e assistência para idosos através de um dispositivo de pulso e uma base transmissora conectada à Internet. As principais funções do dispositovo são o monitoramento de batimentos cardíacos, a detecção de queda e um botão de emergência, que podem ser acessados através de uma solução web.

Palavras-chave: Idoso, monitoramento, wearables, ultra low power, Internet of things (IoT).

\section{DESCRIÇÃO DO PROJETO / CARACTERIZAÇÃO}

\subsection{Objetivo Geral}
Este projeto tem como objetivodesenvolver uma solução não invasiva para o acompanhamento da saúde - principalmente - de idosos, através do monitoramento de batimentos cardíacos, da detecção de possíveis quedas e da presença de um botão de pânico que pode ser acionado em caso de qualquer emergência, além das funções de monitoramento de bateria e de relógio. 

\subsection{Objetivos Específicos}
O projeto consiste em três subsistemas específicos: o dispositivo, a base transmissora e a aplicação web, cada qual com seus obejtivos específicos.

\subsubsection{Dispositivo de pulso}
\begin{enumerate}
\item Desenvolver um firmware para o SoC (System on Chip) Texas CC430, presente no kit de relógio de pulso eZ430-CHRONOS-915.
\item Desenvolver uma função de monitoramento do acelerometro, para medir batimentos cadíacos e possíveis quedas.
\item Desenvolver uma função de monitoramento do botão de emergência, para indicar quando o botão foi acionado.
\item Desenvolver uma função de monitoramento de bateria, para indicar o nível atual e quando uma troca deverá ser feita.
\item Desenvolver as funções anteriores de forma a economizar energia e maximizar a duração da bateria, para reduzir o tempo entre trocas.
\item Desenvolver um protocolo de comunicação com a base transmissora, para receber e enviar dados. 
\end{enumerate}

%O dispositivo de pulso consiste num relógio digital incrementado com sensores para monitorar o batimento cardíaco, as possíveis quedas e o acionamento do %botão de pânico, consumindo o mínimo possível de energia para ser enquadrado como um dispositivo "ultra low power" alimentado por uma bateria de 3.3V de %relógio. O monitoramento deve acontecer em tempo real e reportar - imediatamente - à base transmissora qualquer alteração considerada suspeita. Dentre as %funções operacionais, o dispositivo de pulso deve alertar a base sobre a necessidade da troca iminente da bateria e receber a atualização de informações %relevantes.

\subsubsection{Base transmissora}
Desenvolver um circuito que consistirá no rádio de comunicação com o relógio, um microcontrolador, e uma interface Ethernet. O firmware do microcontrolador deverá se capaz de realizar o comunicação com o relógio e transmitir e receber dados da internet, através da porta Ethernet. A base também deverá contar com um web server capaz de mostrar informações e configurações do relógio e da própria base.


A base transmissora fará o intermédio entre o dispositivo de pulso e a internet. A base deverá receber as informações do dispositivo via rádio-frequência e reportá-las à aplicação de monitoramento web através da internet.

\subsubsection{Aplicação Web}
Desenvoler uma aplicação web, que recebe e envia dados, através de requisições, para a base transmissora. A aplicação, no momento, não contará com interface grafica avançada, mas apenas o necessário para a operação do sistema.

A aplicação deve ser uma plataforma de fácil acompanhamento dos dados recebidos dos dispositivos de pulso, além poder enviar informações aos dispositivos.

\subsection{Diagrama}

\section{JUSTIFICATIVA E RESULTADOS ESPERADOS}

\subsection{Justificativa Resumida}

\subsection{Resultados Esperados}

\subsubsection{Tecnológicos}

\subsubsection{Científicos}

\subsubsection{Econômicos}

\subsubsection{Sociais}

\section{METODOLOGIA E MECANISMOS DE GESTÃO}

\subsection{Metodologia}

\subsubsection{Validação das curvas características dos sensores utilizados}

\subsection{Cronograma Resumido e Datas Importantes}

\subsection{Cronograma Detalhado}

\subsection{Análise de Riscos}

\subsubsection{Topologia de controle inadequada}

\section{DOCUMENTAÇÃO}

\subsection{Estrutura do sumário}

\nocite {Knuth92,ConcreteMath,Simpson,Er01}

\bibliographystyle{amsplain}
\bibliography{refs}

\end{document}